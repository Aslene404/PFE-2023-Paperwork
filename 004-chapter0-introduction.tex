% @Author: Jacem Chaieb
% @Date:   2015-07-26 13:42:06
% @Last Modified by:   Jacem Chaieb
% @Last Modified time: 2016-04-12 15:49:40

%%%%%%%%%%%%%%%%%%%%%%%%%%%%
% CHAPTER                  %
%%%%%%%%%%%%%%%%%%%%%%%%%%%%
\chapter*{General Introduction}
\markboth{\MakeUppercase{General Introduction}}{}%
In today's digital world, where businesses are increasingly relying on cloud-based services and software,such as the Salesforce Platform, efficient management of user accounts has become crucial. This area demonstrates
multiple important investments thanks to the increasing number of interested developers
in this platform who offer an infinity of useful applications.\\
These applications are accessible everywhere and through multiple devices as long as that device is connected to Salesforce, such availability is provided thanks to the robust web and mobile infrastructure of the Salesforce platform.\\
That's why we wanted to create a lightning solution for individuals and enterprises who want to manage and monitor their user's activity within Salesforce and we took community users as a base point. Our application will provide multiple information and statistics about each user as well as enable modifications to such information, it will also provide useful KPI charts and a smart chatbot solution for administrators and community managers.\\
Our end-of-study project entitled "Salesforce Community Management Lightning Application" concludes our summer training as a computer engineer.\\
The project was carried out over six months, within the company TECHLEAD. This report summarizes the stages of realization of this
project. Its purpose is to situate the context of the project, to describe the resulting application, the methods, and tools used as well as the results obtained.\\
This report follows the following organization:\\
The first chapter is entitled "General Framework of the Project", which is an introductory chapter presenting the host company, the problem, the solution
proposed, and the objectives of the project, a study of the existing and the process of
development of our application.\\
The second chapter, "Specification of needs", is used to identify the actors of our application and then to specify the functional and non-functional needs.
functionalities to which our application must respond, making it possible to identify
its main features.\\
The third chapter, "Conception", serves to describe the conceptual diagrams and the architecture applied to our proposed solution.\\
The fourth and final chapter, "Realization", illustrates the realization
of our project through the presentation of the environment and the development tools as well as the visualization of the results of our work through
the main application interfaces.\\
Finally, we end the report with a general conclusion in which
we recapitulate the work carried out and we present the prospects.

